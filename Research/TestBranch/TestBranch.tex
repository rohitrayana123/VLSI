%% <+Last Edited: Fri 14 Feb 2014 19:06:33 GMT by seblovett on seblovett-Ubuntu +>
\documentclass[12pt,a4paper]{article}
\usepackage[nodayofweek]{datetime}
%\usepackage{natbib}
\usepackage{listings}
\usepackage{bibunits}
\title{ELEC6025: VLSI Design Project \\Part 1: Microprocessor Research\\Topic: Test and Branch}
\author{Henry Lovett\\ Team: R4\\Course Tutor: Mr B. Iain McNally}
\date{\today}
\begin{document}
\begin{titlepage}
\maketitle
\end{titlepage}

\tableofcontents
\clearpage
\begin{bibunit}[is-unsrt]
\section{Introduction}
%A short introduction to the material presented in this report \dots\cite{greenwade93}

This report details the research done for the ``Test and Branch'' part of team R4 for the VLSI design module (ELEC6027).
It looks into 2 aspects of processor design - condition codes and conditional branches. 
The report begins with a overview of what should be achieved by the condition codes and branches. 
Three case studies of some simple architectures are then discussed, looking into how they work and includes code snippets. 
Some honourable mentions of other architectures are discussed also. 
The report then finishes with a comparison of the case studies and discusses the conclusions that can be drawn from this research
The authors' recommendation for the implementation is also given in this section.

One or more sections covering your research topic.
Here you should include appropriate figures and code snippets to illustrate your discussion. Ensure that all figures and code snippets are properly explained in your text.
Where text, figures or code snippets are copied from another source, the source must be clearly acknowledged. Copied text must be surrounded by quotation marks "..." to show clearly that it is copied. 

\section{Operation}
What is the point in flags and branches. Why do we need them.

\section{Case studies}

Three case studies are discussed here. 
For each architecture studied, there is a description of the flags implemented (if any) and instructions used for conditional branches. 
Also, two C code snippets, seen in listings \ref{ListC} and \ref{ListC2}, are converted to assembler for each architecture to see how they compare.
\begin{lstlisting}[frame=single,caption=C Code,language=C,label=ListC]
uint16_t a = 0;
for(uint16_t i = 0; i < 10; i++)
{
	a = a + i;
}
\end{lstlisting}
\begin{lstlisting}[frame=single,caption=C Code,language=C,label=ListC2]
if ( a > 0 )
	b = 1;
else
	b = 0;
\end{lstlisting}
\subsection{ARM Cortex M0}

The ARM Cortex M0 is a 32-bit RISC architecture that implements the Thumb/Thumb2 instruction set \cite{ARM:CortexM0}. 
It implements 19 formats of instructions in total, one of which is the `conditional branch' instruction. 
The conditional branch instruction depends on the value in the status register from a previous operation.

\subsubsection{Flags}
The Cortex M0 uses 4 flags; Carry, oVerflow, Zero and Negative \cite{ARM:Flags}. 
The flags are stored in a specific register, the Application Program Status Register (APSR). 
The APSR is updated by a number of instructions. 
All arithmetic instructions have the ability to update the register depending on the result of the arithmetic operation. 
This can be seen in the assembler by an {\textit `S'} suffix to the instruction.
This optional update is only in the Thumb2 instruction set, not the Thumb.
In the Thumb instruction set, all ALU operations update the status register.

There are also four instructions which update the status register - Compare (CMP), Compare Negative (CMN), Test Bits (TST) and Test Equivalence (TEQ).
These can all be part of an ALU instruction, or an immediate instruction \cite{ARM:CMPCMN}.
Compare subtracts the register value from the other operand, and CMN is the same, but negates the second operand.
These two instructions update all the flags, but discard the result of the operation.
TST performs an AND function between the two operands. 
This is the same as an AND instruction, but only affects the Z and N flags.
Similarly, there exists a TEQ instruction which conducts and XOR between the two operands and only updates the Z and N flags \cite{ARM:TSTTEQ}.
The TST and TEQ instructions do not store the result of the operation.


\subsubsection{Conditional Execution}
%ARM also supports conditional execution, however this is not including in the Thumb instruction set (but is in the Thumb2 instruction set).
Thumb2 has support for conditional execution.
This is where operations are only executed if the condition code in the instruction matches up with the value in the status register.
In total, there are 15 condition codes \cite{ARM:conditioncodes} and are similar to the conditional branch options, discussed in section \ref{arm:conditionalbranch}.
This is particularly useful in a pipelined processor for small if-else clauses as no flushing / stalling needs to be done, as the execution of the instruction is done, but the result isn't stored.
A small drop in performance is seen, but this isn't as bad as flushing a deep pipeline.

In the Thumb instruction set, the lack of these condition codes is replaced by the use of the conditional branches.
In an if-else clause, there must be one conditional branch, and one unconditional branch.
Thumb does not implement the condition execution due to it being a 16-bit instruction set - Thumb2 is 32-bit.


\begin{table}
\centering
\begin{tabular}{p{0.15\textwidth}p{0.15\textwidth}p{0.70\textwidth}}\hline
Flag & Shorthand & Explanation \\ \hline
Carry & C & Set on arithmetic carry or borrow\\
Overflow & V & Indicates two's complement overflow\\
Zero & Z & Set if the ALU result is zero\\
Negative & N & Set to the most significant bit of the result (sign). \\
\end{tabular}
\caption{Explanation of the flags in the ARM Cortex M0}
\end{table}
\subsubsection{Conditional Branch}\label{arm:conditionalbranch}

The Thumb instruction set has a conditional branch instruction.
This is capable of incrementing the program counter by an immediate value, depending on the flags in the status register.
Fourteen different conditions are supported, and the full list can be seen in \cite{ARM:Thumb}.
They include individual flag tests, as well as both signed and unsigned greater/less than tests.
The value of the flags is from the last instruction that set the flags - in Thumb2, this is the last instruction with an {\textit 'S'} suffix.
If the branch is decided to be taken, then the program counter is incremented by the immediate value.

In Thumb2, the conditional execution can be used on a branch instruction to conditionally alter the value in the program counter.

\subsubsection{Assembler Examples}
Below are snippets of assembly language for the Cortex M0 to implement the C code seen in listings \ref{ListC} and \ref{ListC2}.
Here, the advantage of the conditional execution can be seen.
In a small if-else clause, it only takes three instructions, and no branches.
This speeds up the processor, as conditional branches can be costly.
They will work very well in a pipelined processor, as well as a non-pipelined processor.

\begin{lstlisting}[frame=single,caption=Thumb/Thumb2 Assembler for code in listing \ref{ListC},label=ARM1]
.def a=$r1
.def i=$r2
MOV a #0 ;load a with 0
MOV i #0 ;load i with 0
loop:
CMP i #10 ;compare i to 10.
BEQ exit ;branch if i == 10
ADDNE a a i ; a = a + i if i != 10
ADDNE i i #1 ; i++ if i != 10
BNE loop ; jump to loop if i != 10
exit: 
...
\end{lstlisting}


\begin{lstlisting}[frame=single,caption=Thumb/Thumb2 Assembler for code in listing \ref{ListC2},label=ARM2]
.def a $r1
.def b $r3
CMP a #0 ; compare a to 0
MOVEQ b #1 ; if a == 0, load b with 1
MOVENE b #0 ; if a != 0, load b with 0
\end{lstlisting}


\subsection{Intel 8086}

The 8086 is a 16 bit microprocessor released in 1978.
This was the original processor designed by Intel.
Is is a CISC based architecture and implements the original x86 instruction set.
Since then, the x86 instruction set has been expanded and the original instructions are still implemented in the lasted Intel CPUs.

\subsubsection{Flags}

The Intel 8086 has 9 flags, detailed in table \ref{tab:Intel:flags}.
The status register is updated when a compare (\texttt{CMP, CMPSB, CMPSW}) instruction is executed, or after an arithmetic operation.


\begin{table}
\centering
\caption{Intel 8086 Flags from \cite{Intel:flags}}
\label{tab:Intel:flags}
\begin{tabular}{p{0.15\textwidth}p{0.15\textwidth}p{0.70\textwidth}}\hline
%\begin{tabular}{ccc} \hline
Flag & Shorthand & Explanation \\ \hline
Carry & C & Carry from the 8 bit arithmetic \\
Zero & Z & Set if the result of the ALU is zero \\
Sign & S & Set to the most significant bit of the ALU result \\
Overflow & O & Set on two's complement overflow \\
Parity & P & 1 if the ALU result is even parity, 0 if odd \cite{flag:p} \\
Auxiliary (or Adjust) & A & The carry generated from the low 4 bits of the ALU operation \cite{flag:a}\\
Trap & T & Used for single step debugging \cite{flag:t} \\
Direction & D & Sets the source index to increment / decrement (0/1) \cite{flag:d} \\
Interrupt & I & Enables / disables interrupts \cite{flag:i} \\
\end{tabular}
\end{table}

\subsubsection{Conditional Branch}

The 8086 has a total of 31 conditional branch instructions, and one jump instruction \cite{Intel:instructionset}.
The conditional branches cover many different logic combinations between 5 flags - C, Z, O, S and P.
This enables the 8086 to make signed and unsigned decisions.
The full set of branch instructions can be seen at \cite{Intel:condjumps}.
There is also a branch instruction that checks the value of a register to be zero, rather than the flags. 

In the assembly language, there are multiple instructions that map to the same instruction.
For example, the \texttt{JE} instruction checks that the \texttt{Z} flag is set, but the \texttt{JZ} instruction also does this.
It can be seen that this is to make the code more readable as the instructions actually map to the same machine code \cite{Intel:datasheet}.
In total, therefore, there are 17 actual instructions.

The 8086 also implements a \texttt{LOOP} instruction.
This is linked to a specific register (usually the \texttt{CX} register) and will branch if the value of \texttt{CX} does not become $0$ after being decremented.
There is another instruction, \texttt{LOOPE / LOOPZ}, that jumps if the counter is zero, and the zero flag is set. 
This has a complementary instruction, \texttt{LOOPNE / LOOPNZ}, which jumps if the counter is nonzero and the zero flag is not set.
These instructions are useful, as it combines the decrement of a counter with a branch, resulting in a quicker operation for loops.


\subsubsection{Assembler Example}
\begin{lstlisting}[frame=single,caption=Intel 8086 assembler for listing \ref{ListC},label=intel1]
.def a ax
.def i cx
mov a,0 ; initialise a
mov i,10 ; initialise i
looplabel: 
add a,a,i ; do the sum
loop looplabel ; decrement cx, if cx != 0, jump to looplabel
\end{lstlisting}

\begin{lstlisting}[frame=single,caption=Intel 8086 assembler for listing \ref{ListC2},label=intel2]
.def a ax
.def b bx
cmp a,0 ; compare a to zero
je else: ; jump if zero flag is set (0 == a)
mov b,0 ; store 0 to b
jmp exit ; have to jump to exit
else:
 move b,1 ; store 1 to b
exit:

\end{lstlisting}


\subsection{MIPS}

MIPS is a 32-bit RISC architecture.
It is one of the early RISC machines and is now owned by Imagination Technologies and is still used today.
The simple design of the MIPS architecture lends itself as a learning tool for computer architecture.

\subsubsection{Flags}

MIPS does not have a status register, and only one flag.
The flag used is a zero flag. 
This means that the conditional branches can only rely on if a register value is zero or not.


\subsubsection{Conditional Branch}

MIPS implements only two branch instructions; branch if equal (BEQ) and branch if not equal (BNE).
With only these two instructions, it is difficult to see how this can be used to fully control the program flow.
There exists an instruction which helps with control flow, but is not a conditional jump - the Set Less Than (SLT) instruction.
SLT stores compares two register values.
A 1 is then stored in the destination register if operand 2 is less than operand 3, else 0 is stored \cite{patterson2013computer}.

This instruction fulfils the lack of branch instructions.
MIPS also has a register which always reads zero, which can then be compared to the value returned by the SLT instruction.
This method results in more instructions needed to be executed to conduct a conditional branch, but results in significantly less instructions needed.



\subsubsection{Assembler Example}
\begin{lstlisting}[frame=single,caption=MIPS assembler for listing \ref{ListC},label=mips1]
.def a $r1
.def i $r2
add a $r0 $r0 ; load a with 0
add i $r0 0 ; load i with 0
loop:
slti $r3 i 10 ; slt with immediate - i < 10
beq $r0 $r3 exit ; if r3 == 0, jump to exit
add a a i ; do the add
jmp loop ; jump back to the beginning of the loop
exit:
\end{lstlisting}
\begin{lstlisting}[frame=single,caption=MIPS assembler for listing \ref{ListC2},label=mips2]
.def a $r1
.def b $r2
slt $r3 $r0 a ; 0 < a
bne $r0 $r3 else ; jump if the result isn't zero
addi b $r0 1 ; load b with 1
jmp exit
else:
addi b $r0 0 ; load b with 0
exit:
\end{lstlisting}


\section{Other Architectures}
\subsection{DEC Alpha}

Architectures that don't use flags are not common. 
The DEC Alpha is another example of an architecture that does not use flags.
The conditional branch instruction can conduct 6 comparisions to zero; equal, not equal, greater than (or equal) and less than (or equal).
The arithmetic for the operation must be done before hand, as in the MIPS.
The conditional branch then can execute by comparing a precomputed value. 


\subsection{AVR}
Even though the AVR is a microcontroller core, it still posed interesting reading. 
The AVR uses a status register and has 8 flags in total \cite{atmel:instructions}.
A number of these are similar to ARM and Intel (C, Z, N, V, S, H, I), but the AVR implements a \texttt{'T'} flag.
The T flag in the AVR is not the same as in the 8086, but is used as a more user defined bit in the status register.
The instruction set allows branching if this flag is set or not, and can be set to a bit value in a byte. 
\cite{avr:asm} has some examples of how this T flag can be used.


\section{Conclusion}
Here I would like you to discuss how the issues raised in your report will affect your processor design. Where you have seen different processors opting for different solutions to the same problem you should discuss their relative merits in the context of your design. 

There is a compromise that must be made with the test and branch aspect of processor design.
The use of flags is advantageous as more complex branches can be conducted depending on the result of the ALU operation.
Also, the branch only takes one instruction to complete the comparison, and one to do the jump, totalling two in the worst case.
However, this then requires more instructions, potentially increasing the size of the opcode field in instruction set.
This also requires more design in the ALU to calculate the flags, as well as a more complex control unit to process the data.

With out flags, the compiler / programmer must conduct operations to reduce the branch problem to an equality, or a less-than. 
Although the example in listings \ref{mips} and \ref{mips2} are simple, this is not always the case. 
A greater/less than or equal comparison will require addition to an operand, before conduction the SLT instruction.
This can lengthen the execution of branches - which are used a lot in programs.
The advantage, however, is that the hardware implementation is much simpler, and the control is smaller.
A compare to zero is needed, as well as an extra ALU operation (SLT) which transfers the borrow signal to the LSB of the result. 

Amdahl's Law says that to increase the performance, the common case must be made fast \cite{amdahl}.
Branches are reasonably common to occur in a program, as the principle of a processor is to be able to dynamically make decisions. 
Performance is a key aspect to processor design and by executing less instructions to do the same task will give an increase in performance.
The recommendation, therefore, is that flags should be used at an expense of more hardware, instructions and complexity of the control.
However, the performance and flexibility of the processor should be advantageous.

\putbib[references]
\end{bibunit}

\begin{bibunit}[is-unsrt]
\renewcommand{\refname}{Bibliography}
\nocite{*}
\putbib[bibliography]
\end{bibunit}
%References - reading referenced in text
%\bibliographystyle{plain}
%\bibliography{references}
%%Biliography - all the reading done
%\renewcommand{\refname}{Bibliography}
%\nocite{*}
%\bibliography{bibliography}
%\begin{thebibliography}{9}
%\bibitem{test} ARM Example, 1992
%
%\end{thebibliography}
\end{document}
