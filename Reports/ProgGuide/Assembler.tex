%  Assembler.tex
%  Document created by seblovett on hind.ecs.soton.ac.uk
%  Date created: Thu 27 Mar 2014 10:13:13 GMT
%  <+Last Edited: Sun 13 Apr 2014 21:33:17 BST by seblovett on seblovett-Ubuntu +>

\section{Assembler}
The current instruction set architecture includes an assembler for converting symbolic sequences into machine code. This chapter outlines the required formatting and available features of this assembler. 

\subsection{Instruction Formatting}
Each instruction must be formatted using the following syntax, here "[\dots]" indicates an optional field:

\begin{center}\texttt{[.LABELNAME] MNEMONIC, OPERANDS, ..., :[COMMENTS]}\end{center}

\begin{center}eg. \texttt{.loop ADDI, R5, R3, \#5 :Add 5 to R3}
\end{center}

\noindent Comments may be added by preceding them with either : or ;.\\

\noindent Accepted general purpose register values are: R0, R1, R2, R3, R4, R5, R6, R7, SP. These can be upper or lower case and SP is equivalently evaluated to R7.\\

\noindent Branch instructions can take either a symbolic or numeric value. 
Where a numeric must be relative and between -32 and 31 for a JMP instruction, or between -128 and 127 for any other branch type. 
If the branch exceeds the accepted range, the assembler will flag an error message. \\

\noindent All label names must begin with a '.' while .ISR/.isr and .define are special cases used for the interrupt service routine and variable definitions respectively. \\

\noindent Instruction-less or comments only lines are allowed within the assembly file. \\
\newpage
\noindent {\bf Special Case Label}

\noindent The .ISR/.isr label is reserved for the interrupt service routine and may be located anywhere within the file but must finish with a 'RETI' instruction and be no longer than 126 lines of code. Branches may occur within the ISR, but are not allowed into this subroutine with the exception of a return from a separate subroutine.\\

\subsection{Assembler Directives}
%There is one supported assembler directive  for assigning meaningful names to each of the general purpose registers. 
Symbolic label names are supported for branch-type instructions. Following the previous syntax definition for '.LABELNAME', they can be used instead of numeric branching provided they branch no further than the maximum distance allowed for the instruction used. 
Definitions are supported by the assembler. 
They are used to assign meaningful names to the GPRs to aid with programming.
Definitions can occur at any point within the file and create a mapping from that point onwards. 
Different names can be assigned to the same register, but only one is valid at a time. \\


\noindent The accepted syntax for definitions is:

\begin{center}\texttt{.define NAME REGISTER}\end{center}

\subsection{Running The Assembler}
The assembler reads a `.asm' file and outputs a `.hex' file in hexadecimal format. 
It is run by typing ``./assemble filename'' at the command line when in the directory of both the assembler executable and the program assembly file. ``filename'' does not have to include the .asm file extension. 
The outputted file is saved to the same directory as the input file. \\
\todo[inline]{HSL: I'm going to add an option parser to make the UI a bit easier. This section is likely to change a fair amount}
\noindent Typing -h or --help instead of the file name will bring up the help menu with version information and basic formatting support. 

\newpage
\subsection{Error Messages}
\begin{center}
	\centering
	\begin{tabular}{r|p{12cm}}
		\multicolumn{1}{c}{\bf Code} & \multicolumn{1}{c}{\bf Description} \\
		\hline\hline
		ERROR1& Instruction mneumonic is not recognized \\
		ERROR2& Register code within instruction is not recognized\\
		ERROR3& Branch condition code is not recognised\\
		ERROR4& Attempting to branch to undefined location \\
		ERROR5& Instruction mneumonic is not recognized \\
		ERROR6& Attempting to shift by more than 16 or perform a negative shift \\
		ERROR7& Magnitude of immediate value for ADDI, ADCI, SUBI, SUCI, LDW or STW is too large\\
		ERROR8& Magnitude of immediate value for CMPI or JMP is too large \\
		ERROR9& Magnitude of immediate value for ADDIB, SUBIB, LUI or LLI is too large \\
		ERROR10& Attempting to jump more than 127 forward or 128 backwards \\
		ERROR11& Duplicate symbolic link names \\
		ERROR12& Illegal branch to ISR \\
		ERROR13& Multiple ISRs in file \\
		ERROR14& Invalid formatting for .define directive \\
	\end{tabular}
\end{center}
