%  InstructionSet.tex
%  Date created: Thu 27 Mar 2014
%  <+Last edited: Sat 05 apr 2014 by mw20g10

\newpage
\section{Instruction Set}

The complete instruction set architecture includes a number of instructions for performing calculations on data, moving data between external memory and general purpose registers, transfer of control within a program and interrupt handling. 
It is based around a RISC architecture and as such has a highly orthogonal formatting of bit fields within the instruction code. 

All instruction implemented by the architecture fall into one of 6 groups:
\begin{itemize}
	\item Data Manipulation
	\item Byte Immediate
	\item Data Transfer
	\item Control Transfer
	\item Stack Operations
	\item Interrupts
\end{itemize}

Each instruction has only one addressing mode associated with it, determined by which group it fall within. Data manipulation instructions have either a register register or register immediate addressing mode for performing arithmetic, logical and shifting type operations. Byte immediate instructions have a register immediate addressing mode for arithmetic and load immediate type operations. Data transfer instructions have a base plus offset addressing mode for accessing external memory using an address stored in a GPR. Control transfer instructions have PC relative, register indirect and base plus offset addressing modes for changing the value of the program counter. Stack operations have register indirect preincrement or register indirect postdecrement addressing modes for accessing external memory and adjusting the stack pointer value. While interrupt operations have register indirect with postdecrement or preincrement addressing modes for restoring program counter and accessing the stack.
\newpage

\noindent{\bf General Instruction Formatting}
%\newcolumntype{B}{>{\begin{varwidth}{0.2cm}} c <{\end{varwidth}}}
\newcolumntype{B}{c}
\begin{table}[h]
\centering
\footnotesize
\setlength{\tabcolsep}{2.5pt}
\makebox[\linewidth]{
\begin{tabular}{|r|l|l||BBBBBBBBBBBBBBBc|}
	 \multicolumn{1}{r}{} & \multicolumn{1}{l}{\bf Instruction Type} & \multicolumn{1}{l}{\bf Sub-Type} & 15 & 14 & 13 & 12 & 11 & 10 & 9 & 8 & 7 & 6 & 5 & 4 & 3 & 2 & 1 & \multicolumn{1}{B}{0} \\
	\hline
	A1 & \multirow{2}{*}{\bf Data Manipulation} & {\bf Register} & \multicolumn{5}{B|}{\multirow{2}{*}{Opcode}} & \multicolumn{3}{B|}{Rd} & \multicolumn{3}{B|}{Ra} & \multicolumn{3}{B|}{Rb} & X & X \\
	\cline{1-1} \cline{3-3} \cline{9-19}
	A2 &  & {\bf Immediate} & \multicolumn{5}{B|}{} & \multicolumn{3}{B|}{Rd} & \multicolumn{3}{B|}{Ra} & \multicolumn{5}{B|}{imm4/5} \\
	\hline
	B & \multicolumn{2}{l||}{\bf Byte Immediate} & \multicolumn{5}{B|}{Opcode} & \multicolumn{3}{B|}{Rd} & \multicolumn{8}{B|}{imm8} \\
	\hline
	C & \multicolumn{2}{l||}{\bf Data Transfer} & 0 & \multicolumn{1}{|B|}{LS} & 0 & 0 & \multicolumn{1}{B|}{0} & \multicolumn{3}{B|}{Rd}  &\multicolumn{3}{B|}{Ra} & \multicolumn{5}{B|}{imm5} \\
	\hline
	D1 & \multirow{2}{*}{\bf Control Transfer} & {\bf Others} & \multirow{2}{*}{1} & \multirow{2}{*}{1} & \multirow{2}{*}{1} & \multirow{2}{*}{1} & \multicolumn{1}{B|}{\multirow{2}{*}{0}} & \multicolumn{3}{B|}{\multirow{2}{*}{Cond.}}  & \multicolumn{8}{B|}{imm8} \\
	\cline{1-1} \cline{3-3} \cline{12-19}
	D2 &  & {\bf Jump} &  &  &  &  & \multicolumn{1}{B|}{} & & & \multicolumn{1}{B|}{ } & \multicolumn{3}{B|}{Ra} & \multicolumn{5}{B|}{imm5} \\
	\hline
	E & \multicolumn{2}{l||}{\bf Stack Operations} & 0 & \multicolumn{1}{|B|}{U} & 0 & 0 & \multicolumn{1}{B|}{1} & \multicolumn{1}{B|}{L} & X & \multicolumn{1}{B|}{X} & \multicolumn{3}{B|}{Ra} & 0 & 0 & 0 & 0 & \multicolumn{1}{B|}{1} \\
	\hline
	F & \multicolumn{2}{l||}{\bf Interrupts} & 1 & 1 & 0 & 0 & \multicolumn{1}{B|}{1} & \multicolumn{3}{B|}{ICond.} & 1 & 1 & \multicolumn{1}{B|}{1} & X & X & X & X & \multicolumn{1}{B|}{X} \\
	\hline
\end{tabular}
}
\end{table}
\hspace{0pt}\\\\

\begingroup
\setlength{\abovedisplayskip}{0pt}
\noindent{\bf Instruction Field Definitions} \\
\begin{alignat*}{2}
	\text{Opcode:}& \text{ Operation code as defined for each instruction} \\
	\text{Rd:}& \text{ Destination Register} \\
	\text{Ra:}& \text{ Source register 1} \\
	\text{Rb:}& \text{ Source register 2} \\
	\text{immX:}& \text{ Immediate value of length X} \\
	\text{Cond.:}& \text{ Branching condition code as defined for branch instructions} \\
	\text{ICond.:}& \text{ Interrupt instruction code as defined for interrupt instructions} \\
	\text{LS:}& \text{ 0=Load Data, 1=Store Data} \\
	\text{U:}& \text{ 1=PUSH, 0=POP} \\
	\text{L:}& \text{ 1=Use Link Register, 0=Use General Purpose Register} \\
\end{alignat*}
\endgroup

\newpage
\noindent{\bf Pseudocode Notation}
\begin{table}[h]
\centering
\footnotesize
\makebox[\linewidth]{
\begin{tabular}{|c|l|}
	\hline
	{\bf Symbol} & \multicolumn{1}{c|}{\bf Meaning} \\\hline
	$\leftarrow$, $\rightarrow$ & Assignment \\\hline
	Ra[{\itshape x}] & Bit {\itshape x} of register Ra \\\hline
	Ra[{\itshape x} : {\itshape y}] & Bit range from {\itshape x} to {\itshape y} of register Ra \\\hline
	$+Ra$ & Positive value in Register Ra \\\hline
	$-Ra$ & Negative value in Register Ra \\\hline
	\textless & Numerically greater than \\\hline
	\textgreater & Numerically less than \\\hline
	\textless\textless & Logical shift left \\\hline
	\textgreater\textgreater & Logical shift right \\\hline
	\textgreater\textgreater\textgreater & arithmetic shift right \\\hline
	Mem[{\itshape val}] & Data at memory location with address {\itshape val} \\\hline
	\{{\itshape x}, {\itshape y}\} & Contatenation of {\itshape x} and {\itshape y} to form a 16-bit value \\\hline
	({\itshape cond})? & Operation performed if {\itshape cond} evaluates to true \\\hline
	! & Bitwise Negation \\\hline
\end{tabular}
}
\end{table}\\


%
% Instructions 1-8
%
\Imnemonic{Add Word}{ADD}
\Iformat{A}{00010}
\Isyntax{ADD Rd, Ra, Rb}{ADD R5, R3, R2}
\Ioperation{Rd $\leftarrow$}{Ra + Rb}{C}{V}{b}{0}{N}{Z}
\Idesc{The 16-bit word in GPR Ra is added to the 16-bit word in GPR Rb and the result is placed into GPR Rd. \\\\ The addressing mode of this instruction is Register Register.}
\newpage
\Imnemonic{Add Immediate}{ADDI}
\Iformat{a}{00110}
\Isyntax{ADDI Rd, Ra, \#imm5}{ADDI R5, R3, \#7}
\Ioperation{Rd $\leftarrow$}{Ra + \#imm5}{C}{V}{5}{0}{N}{Z}
\Idesc{The 16-bit word in GPR Ra is added to the sign-extended 5-bit value given in the instruction and the result is placed into GPR Rd. \\\\ The addressing mode of this instruction is Register Immediate.}
\newpage
\Imnemonic{Add Immediate Byte}{ADDIB}
\Iformat{B}{00011}
\Isyntax{ADDIB Rd, \#imm8}{ADDIB R5, \#93}
\Ioperation{Rd $\leftarrow$}{Rd + \#imm8}{C}{V}{8}{0}{N}{Z}
\Idesc{The 16-bit word in GPR Rd is added to the sign-extended 8-bit value given in the instruction and the result is placed into GPR Rd. \\\\ The addressing mode of this instruction is Register Immediate.}
\newpage
\Imnemonic{Add Word With Carry}{ADC}
\Iformat{A}{00100}
\Isyntax{ADC Rd, Ra, Rb}{ADC R5, R3, R2}
\Ioperation{Rd $\leftarrow$}{Ra + Rb + C}{C}{V}{b}{c}{N}{Z}
\Idesc{The 16-bit word in GPR Ra is added to the 16-bit word in GPR Rb with the added carry in set according to the Carry flag from previous operation, and the result is placed into GPR Rd. \\\\ The addressing mode of this instruction is Register Register.}
\newpage
\Imnemonic{Add Immediate With Carry}{ADCI}
\Iformat{a}{00101}
\Isyntax{ADCI Rd, Ra, \#imm5}{ADCI R5, R4, \#7}
\Ioperation{Rd $\leftarrow$}{Ra + \#imm5 + C}{C}{V}{5}{c}{N}{Z}
\Idesc{The 16-bit word in GPR Ra is added to the sign-extended 5-bit value given in the instruction with carry in set according to the Carry flag from previous operation, and the result is placed into GPR Rd. \\\\ The addressing mode of this instruction is Register Immediate.}
\newpage
\Imnemonic{Negate Word}{NEG}
\Iformat{A}{11010}
\Isyntax{NEG Rd, Ra}{NEG R5, R3}
\Ioperation{Rd $\leftarrow$}{0 - Ra}{C}{V}{0}{0}{N}{Z}
\Idesc{The 16-bit word in GPR Ra is added to the 16-bit word in GPR Rb and the result is placed into GPR Rd. \\\\ The addressing mode of this instruction is Register Register.}
\newpage
\Imnemonic{Subtract Word}{SUB}
\Iformat{A}{01010}
\Isyntax{SUB Rd, Ra, Rb}{SUB R5, R3, R2}
\Ioperation{Rd $\leftarrow$}{Ra - Rb}{C}{V}{b}{0}{N}{Z}
\Idesc{The 16-bit word in GPR Rb is subtracted from the 16-bit word in GPR Ra and the result is placed into GPR Rd. \\\\ The addressing mode of this instruction is Register Register.}
\newpage
\Imnemonic{Subtract Immediate}{SUBI}
\Iformat{a}{01110}
\Isyntax{SUBI Rd, Ra, \#imm5}{SUBI R5, R3, \#7}
\Ioperation{Rd $\leftarrow$}{Ra - \#imm5}{C}{V}{5}{0}{N}{Z}
\Idesc{The sign extended 5-bit value given in the instruction is subtracted from the 16-bit word in GPR Ra and the result is placed into GPR Rd. \\\\ The addressing mode of this instruction is Register Immediate.}
\newpage
%
% Instructions 9-16
%
\Imnemonic{Subtract Immediate Byte}{SUBIB}
\Iformat{B}{01011}
\Isyntax{SUBIB Rd, \#imm8}{SUBIB R5, \#93}
\Ioperation{Rd $\leftarrow$}{Rd - \#imm8}{C}{V}{8}{0}{N}{Z}
\Idesc{The 8-bit immediate value given in the instruction is subtracted from the 16-bit word in GPR Rd and the result is placed into GPR Rd. \\\\ The addressing mode of this instruction is Register Immediate.}
\newpage
\Imnemonic{Subtract Word With Carry}{SUC}
\Iformat{A}{01100}
\Isyntax{SUC Rd, Ra, Rb}{SUC R5, R3, R2}
\Ioperation{Rd $\leftarrow$}{Ra - Rb - C}{C}{V}{b}{n}{N}{Z}
\Idesc{The 16-bit word in GPR Rb is subtracted from the 16-bit word in GPR Rb with the subtracted carry in set according to the Carry flag from previous operation, and the result is placed into GPR Rd. \\\\ The addressing mode of this instruction is Register Register.}
\newpage
\Imnemonic{Subtract Immediate With Carry}{SUCI}
\Iformat{a}{01101}
\Isyntax{SUCI Rd, Ra, \#imm5}{SUCI R5, R4, \#7}
\Ioperation{Rd $\leftarrow$}{Ra - \#imm5 - C}{C}{V}{5}{n}{N}{Z}
\Idesc{The 5-bit immediate value in instruction is subtracted from the 16-bit word in GPR Ra with the subtracted carry in set according to the Carry flag from previous operation, and the result is placed into GPR Rd. \\\\ The addressing mode of this instruction is Register Immediate.}
\newpage
\Imnemonic{Compare Word}{CMP}
\Iformat{A}{00111}
\Isyntax{CMP Ra, Rb}{CMP R3, R2}
\Ioperation{X}{Ra - Rb}{C}{V}{b}{0}{N}{Z}
\Idesc{The 16-bit word in GPR Rb is subtracted from the 16-bit word in GPR Ra and the status flags are updated without saving the result. \\\\ The addressing mode of this instruction is Register Register.}
\newpage
\Imnemonic{Compare Immediate}{CMPI}
\Iformat{a}{01111}
\Isyntax{CMPI Ra, \#imm5}{CMPI R3, \#7}
\Ioperation{X}{Ra - \#imm5}{C}{V}{5}{0}{N}{Z}
\Idesc{The sign extended 5-bit value given in the instruction is subtracted from the 16-bit word in GPR Ra and the status flags are updated without saving the result. \\\\ The addressing mode of this instruction is Register Immediate.}
\newpage
\Imnemonic{Logical AND}{AND}
\Iformat{A}{10000}
\Isyntax{AND Rd, Ra, Rb}{AND R5, R3, R2}
\Ioperation{Rd $\leftarrow$}{Ra AND Rb}{0}{0}{0}{0}{0}{0}
\Idesc{The logical AND of the 16-bit words in GPRs Ra and Rb is performed and the result is placed into GPR Rd. \\\\ The addressing mode of this instruction is Register Register.}
\newpage
\Imnemonic{Logical OR}{OR}
\Iformat{A}{10001}
\Isyntax{OR Rd, Ra, Rb}{OR R5, R3, R2}
\Ioperation{Rd $\leftarrow$}{Ra OR Rb}{0}{0}{0}{0}{0}{0}
\Idesc{The logical OR of the 16-bit words in GPRs Ra and Rb is performed and the result is placed into GPR Rd. \\\\ The addressing mode of this instruction is Register Register.}
\newpage
\Imnemonic{Logical XOR}{XOR}
\Iformat{A}{10011}
\Isyntax{XOR Rd, Ra, Rb}{XOR R5, R3, R2}
\Ioperation{Rd $\leftarrow$}{Ra XOR Rb}{0}{0}{0}{0}{0}{0}
\Idesc{The logical XOR of the 16-bit words in GPRs Ra and Rb is performed and the result is placed into GPR Rd. \\\\ The addressing mode of this instruction is Register Register.}
\newpage
%
% Instructions 17-24
%
\Imnemonic{Logical NOT}{NOT}
\Iformat{A}{10010}
\Isyntax{NOT Rd, Ra}{NOT R5, R3}
\Ioperation{Rd $\leftarrow$}{NOT Ra}{0}{0}{0}{0}{0}{0}
\Idesc{The logical NOT of the 16-bit word in GPR Ra is performed and the result is placed into GPR Rd. \\\\ The addressing mode of this instruction is Register Register.}
\newpage
\Imnemonic{Logical NAND}{NAND}
\Iformat{A}{10110}
\Isyntax{NAND Rd, Ra, Rb}{NAND R5, R3, R2}
\Ioperation{Rd $\leftarrow$}{Ra NAND Rb}{0}{0}{0}{0}{0}{0}
\Idesc{The logical NAND of the 16-bit words in GPRs Ra and Rb is performed and the result is placed into GPR Rd. \\\\ The addressing mode of this instruction is Register Register.}
\newpage
\Imnemonic{Logical NOR}{NOR}
\Iformat{A}{10111}
\Isyntax{NOR Rd, Ra, Rb}{NOR R5, R3, R2}
\Ioperation{Rd $\leftarrow$}{Ra NOR Rb}{0}{0}{0}{0}{0}{0}
\Idesc{The logical NOR of the 16-bit words in GPRs Ra and Rb is performed and the result is placed into GPR Rd. \\\\ The addressing mode of this instruction is Register Register.}
\newpage
\Imnemonic{Logical Shift Left}{LSL}
\Iformat[s]{a}{11111}
\Isyntax{LSL Rd, Ra, \#imm4}{LSL R5, R3, \#7}
\Ioperation{Rd $\leftarrow$}{Ra $<<$ \#imm4}{0}{0}{0}{0}{0}{0}
\Idesc{The 16-bit word in GPR Ra is shifted left by the 4-bit amount specified in the instruction, shifting in zeros, and the result is placed into GPR Rd. \\\\ The addressing mode of this instruction is Register Immediate.}
\newpage
\Imnemonic{Logical Shift Right}{LSR}
\Iformat[s]{a}{11101}
\Isyntax{LSR Rd, Ra, \#imm4}{LSR R5, R3, \#7}
\Ioperation{Rd $\leftarrow$}{Ra $>>$ \#imm4}{0}{0}{0}{0}{0}{0}
\Idesc{The 16-bit word in GPR Ra is shifted right by the 4-bit amount specified in the instruction, shifting in zeros, and the result is placed into GPR Rd. \\\\ The addressing mode of this instruction is Register Immediate.}
\newpage
\Imnemonic{Arithmetic Shift Right}{ASR}
\Iformat[s]{a}{11100}
\Isyntax{ASR Rd, Ra, \#imm4}{ASR R5, R3, \#7}
\Ioperation{Rd $\leftarrow$}{Ra $>>>$ \#imm4}{0}{0}{0}{0}{0}{0}
\Idesc{The 16-bit word in GPR Ra is shifted right by the 4-bit amount specified in the instruction, shifting in the sign bit of Ra, and the result is placed into GPR Rd. \\\\ The addressing mode of this instruction is Register Immediate.}
\newpage
\Imnemonic{Load Word}{LDW}
\Iformat{C}{0}
\Isyntax{LDW Rd, [Ra, \#imm5]}{LDW R5, [R3, \#7]}
\Ioperation{Rd $\leftarrow$}{Mem[Ra + \#imm5]}{0}{0}{0}{0}{0}{0}
\Idesc{Data is loaded from memory at the resultant address from addition of GPR Ra and the 5-bit immediate value specified in the instruction, and the result is placed into GPR Rd. \\\\ The addressing mode of this instruction is Base Plus Offset.}
\newpage
\Imnemonic{Store Word}{STW}
\Iformat{C}{1}
\Isyntax{STW Rd, [Ra, \#imm5]}{STW R5, [R3, \#7]}
\Ioperation{Rd $\rightarrow$}{Mem[Ra + \#imm5]}{0}{0}{0}{0}{0}{0}
\Idesc{Data in GPR Rd is stored to memory at the resultant address from addition of GPR Ra and the 5-bit immediate value specified in the instruction. \\\\ The addressing mode of this instruction is Base Plus Offset.}
\newpage
%
% Instructions 25-32
%
\Imnemonic{Load Upper Immediate}{LUI}
\Iformat{B}{10100}
\Isyntax{LUI Rd \#imm8}{LUI R5, \#93}
\Ioperation{Rd $\leftarrow$}{\{\#imm8,~0\}}{0}{0}{0}{0}{0}{0}
\Idesc{The 8-bit immediate value provided in the instruction is loaded into the top half in GPR Rd, setting the bottom half to zero. \\\\ The addressing mode of this instruction is Register Immediate.}
\newpage
\Imnemonic{Load Lower Immediate}{LLI}
\Iformat{B}{10101}
\Isyntax{LLI Rd \#imm8}{LLI R5, \#93}
\Ioperation{Rd $\leftarrow$}{\{Rd[15:8],~\#imm8\}}{0}{0}{0}{0}{0}{0}
\Idesc{The 8-bit immediate value provided in the instruction is loaded into the bottom half in GPR Rd, leaving the top half unchanged. \\\\ The addressing mode of this instruction is Register Immediate.}
\newpage
\Imnemonic{Branch Always}{BR}
\Iformat{D}{000}
\Isyntax{BR LABEL}{BR .loop}
\Ioperation{PC $\leftarrow$}{PC + \#imm8}{0}{0}{0}{0}{0}{0}
\Idesc{Unconditionally branch to the resultant address from addition of PC and the 8-bit immediate value specified in the instruction. LABEL can be both a symbolic name or a numeric value, and is capable of jumping forwards or backwards.\\\\ The addressing mode of this instruction is PC Relative.}
\newpage
\Imnemonic{Branch If Not Equal}{BNE}
\Iformat{D}{110}
\Isyntax{BNE LABEL}{BNE .loop}
\Ioperation{PC $\leftarrow$}{PC + \#imm8~(z==0)?}{0}{0}{0}{0}{0}{0}
\Idesc{Conditionally branch to the resultant address from addition of PC and the 8-bit immediate value specified in the instruction if zero status flag (Z) equals zero. LABEL can be both a symbolic name or a numeric value, and is capable of jumping forwards or backwards.\\\\ The addressing mode of this instruction is PC Relative.}
\newpage
\Imnemonic{Branch If Equal}{BE}
\Iformat{D}{111}
\Isyntax{BE LABEL}{BE .loop}
\Ioperation{PC $\leftarrow$}{PC + \#imm8~(z==1)?}{0}{0}{0}{0}{0}{0}
\Idesc{Conditionally branch to the resultant address from addition of PC and the 8-bit immediate value specified in the instruction if zero status flag (Z) equals one. LABEL can be both a symbolic name or a numeric value, and is capable of jumping forwards or backwards.\\\\ The addressing mode of this instruction is PC Relative.}
\newpage
\Imnemonic{Branch If Less Than}{BLT}
\Iformat{D}{100}
\Isyntax{BLT LABEL}{BLT .loop}
\Ioperation{PC $\leftarrow$}{PC + \#imm8~(n\&!v~OR~!n\&v)?}{0}{0}{0}{0}{0}{0}
\Idesc{Conditionally branch to the resultant address from addition of PC and the 8-bit immediate value specified in the instruction if negative status flag and overflow status flag are not equivalent. LABEL can be both a symbolic name or a numeric value, and is capable of jumping forwards or backwards.\\\\ The addressing mode of this instruction is PC Relative.}
\newpage
\Imnemonic{Branch If Greater Than Or Equal}{BGE}
\Iformat{D}{101}
\Isyntax{BGE LABEL}{BGE .loop}
\Ioperation{PC $\leftarrow$}{PC + \#imm8~(n\&v~OR~!n\&!v)?}{0}{0}{0}{0}{0}{0}
\Idesc{Conditionally branch to the resultant address from addition of PC and the 8-bit immediate value specified in the instruction if negative status flag and overflow status flag are equivalent. LABEL can be both a symbolic name or a numeric value, and is capable of jumping forwards or backwards.\\\\ The addressing mode of this instruction is PC Relative.}
\newpage
\Imnemonic{Branch With Link}{BWL}
\Iformat{D}{011}
\Isyntax{BWL LABEL}{BWL .loop}
\Ioperation{LR $\leftarrow$}{PC + 1;~PC $\leftarrow$ PC + \#imm8}{0}{0}{0}{0}{0}{0}
\Idesc{Save the current program counter (PC) value plus one to the link register. Then unconditionally branch to the resultant address from addition of PC and the 8-bit immediate value specified in the instruction. LABEL can be both a symbolic name or a numeric value, and is capable of jumping forwards or backwards.\\\\ The addressing mode of this instruction is PC Relative.}
\newpage
%
% Instructions 33-40
%
\Imnemonic{Return}{RET}
\Iformat{D}{010}
\Isyntax{RET}{RET}
\Ioperation{PC $\leftarrow$}{LR}{0}{0}{0}{0}{0}{0}
\Idesc{Unconditionally branch to the address stored in the link register (LR).\\\\ The addressing mode of this instruction is Register Indirect.}
\newpage
\Imnemonic{Jump}{JMP}
\Iformat{D}{001}
\Isyntax{JMP Ra, \#imm5}{JMP R3, \#7}
\Ioperation{PC $\leftarrow$}{Ra + \#imm5}{0}{0}{0}{0}{0}{0}
\Idesc{Unconditionally jump to the resultant address from the addition of GPR Ra and the 5-bit immediate value specified in the instruction.\\\\ The addressing mode of this instruction is Base Plus Offset.}
\newpage
\Imnemonic{Push From Stack}{PUSH}
\Iformat{E}{1L0}
\Isyntax[M]{PUSH Ra}{PUSH R3}
\Isyntax[F]{PUSH RL}{PUSH RL}
\Ioperation{reg $\rightarrow$}{Mem[R7]; R7 $\leftarrow$ R7 - 1}{0}{0}{0}{0}{0}{0}
\Idesc{`reg' corresponds to either a GPR or the link register, the contents of which are stored to the stack using the address stored in the stack pointer (R7). Then Decrement the stack pointer by one. \\\\The addressing mode of this instruction is Register Indirect Postdecrement.}
\newpage
\Imnemonic{Pop From Stack}{POP}
\Iformat{E}{0L}
\Isyntax[M]{POP Ra}{POP R3}
\Isyntax[F]{POP RL}{POP RL}
\Ioperation{R7 $\leftarrow$}{R7 + 1; Mem[R7] $\leftarrow$ reg;}{0}{0}{0}{0}{0}{0}
\Idesc{Increment the stack pointer by one. Then `reg' corresponds to either a GPR or the link register, the contents of which are retrieved from the stack using the address stored in the stack pointer (R7). \\\\The addressing modes of this instruction are Register Indirect Preincrement.}
\newpage
\Imnemonic{Return From Interrupt}{RETI}
\Iformat{F}{000}
\Isyntax{RETI}{RETI}
\Ioperation{PC $\leftarrow$}{Mem[R7]}{0}{0}{0}{0}{0}{0}
\Idesc{Restore program counter to its value before interrupt occured, which is stored on the stack, pointed to be the stack pointer (R7). This must be the last instruction in an interrupt service routine. \\\\ The addressing mode of this instruction is Register Indirect.}
\newpage
\Imnemonic{Enable Interrupts}{ENAI}
\Iformat{F}{001}
\Isyntax{ENAI}{ENAI}
\Ioperation{IntEn}{Flag $\leftarrow$ 1}{0}{0}{0}{0}{0}{0}
\Idesc{Turn on interrupts by setting interrupt enable flag to true (1).}
\newpage
\Imnemonic{Disable Interrupts}{DISI}
\Iformat{F}{010}
\Isyntax{DISI}{DISI}
\Ioperation{IntEn}{Flag $\leftarrow$ 0}{0}{0}{0}{0}{0}{0}
\Idesc{Turn off interrupts by setting interrupt enable flag to false (0).}
\newpage
\Imnemonic{Store Status Flags}{STF}
\Iformat{F}{011}
\Isyntax{STF}{STF}
\Ioperation{Mem}{[R7] $\leftarrow$ \{12-bit 0, Z, C, V, N\}; R7 $\leftarrow$ R7 - 1;}{0}{0}{0}{0}{0}{0}
\Idesc{Store contents of status flags to stack using address held in stack pointer (R7). Then decrement the stack pointer (R7) by one. \\\\ The addressing more of this instruction is Register Indirect Postdecrement.}
\newpage
%
% Instruction 41
%
\Imnemonic{Load Status Flags}{LDF}
\Iformat{F}{100}
\Isyntax{LDF}{LDF}
\Ioperation{R7 $\leftarrow$}{R7 + 1; \{Z, C, V, N\} $\leftarrow$ Mem[R7][3:0]}{0}{0}{0}{0}{0}{0}
\Idesc{Increment the stack pointer (R7) by one. Then load content of status flags with lower 4 bits of value retrieved from stack using address held in stack pointer (R7). \\\\The addressing more of this instruction is Register Indirect Preincrement.}
\newpage