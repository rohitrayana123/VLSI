%  definitions.tex
%  Document created by seblovett on seblovett-Ubuntu
%  Date created: Thu 24 Apr 2014 18:25:35 BST
%  <+Last Edited: Mon 28 Apr 2014 21:16:02 BST by seblovett on seblovett-Ubuntu +>

% This contains macros and definitions used in the report. 

\definecolor{mygreen}{rgb}{0,0.6,0}
\definecolor{mygray}{rgb}{0.5,0.5,0.5}
\definecolor{mymauve}{rgb}{0.58,0,0.82}

%Code Styles
\lstset{basicstyle=\scriptsize\ttfamily,
  backgroundcolor=\color{white},   % choose the background color; you must add \usepackage{color} or \usepackage{xcolor}
  basicstyle=\footnotesize,        % the size of the fonts that are used for the code
  breakatwhitespace=false,         % sets if automatic breaks should only happen at whitespace
  breaklines=true,                 % sets automatic line breaking
  captionpos=t,                    % sets the caption-position to bottom
  commentstyle=\color{mygreen},    % comment style
  deletekeywords={...},            % if you want to delete keywords from the given language
  escapeinside={\%*}{*)},          % if you want to add LaTeX within your code
  extendedchars=true,              % lets you use non-ASCII characters; for 8-bits encodings only, does not work with UTF-8
  frame=single,                    % adds a frame around the code
  keepspaces=true,                 % keeps spaces in text, useful for keeping indentation of code (possibly needs columns=flexible)
  numbers=left,                    % where to put the line-numbers; possible values are (none, left, right)
  numbersep=5pt,                   % how far the line-numbers are from the code
  numberstyle=\tiny\color{mygray}, % the style that is used for the line-numbers
  rulecolor=\color{black},         % if not set, the frame-color may be changed on line-breaks within not-black text (e.g. comments (green here))
  showspaces=false,                % show spaces everywhere adding particular underscores; it overrides 'showstringspaces'
  showstringspaces=false,          % underline spaces within strings only
  showtabs=false,                  % show tabs within strings adding particular underscores
  stepnumber=1,                    % the step between two line-numbers. If it's 1, each line will be numbered
  tabsize=2,                       % sets default tabsize to 2 spaces
  title=\lstname                   % show the filename of files included with \lstinputlisting; also try caption instead of title
}
\lstdefinestyle{C} {
  language=C,
  otherkeywords={uint16_t,uint32_t,uint8_t},
  stringstyle=\color{mymauve},     % string literal style
  keywordstyle=\color{blue},      % keyword style
	tabsize=4,
}
\lstdefinestyle{sverilog} {
  language=Verilog,
  otherkeywords={always\_ff,always\_comb,assert,logic,return,\$random,\#*},            % if you want to add more keywords to the set
  stringstyle=\color{mymauve},     % string literal style
  keywordstyle=\color{blue}      % keyword style
}
\lstdefinelanguage{samurai}{
	keywords={ADD,ADDI,ADDIB,ADC,ADCI,NEG,SUB,SUBIB,SUC,SUCI,CMP,CMPI,AND,OR,XOR,NOT,NAND,NOR,LSL,LSR,ASR,LDW,STW,LUI,LLI,BR,BNE,BE,BLT,BGE,BWL,RET,JMP,PUSH,POP,RETI,ENAI,DISI,STF,LDF},         % if you want to add more keywords to the se
	comment=[l]{;}
}


\lstdefinestyle{asm} {
  %otherkeywords={ADD,ADDI,ADDIB,ADC,ADCI,NEG,SUB,SUBIB,SUC,SUCI,CMP,CMPI,AND,OR,XOR,NOT,NAND,NOR,LSL,LSR,ASR,LDW,STW,LUI,LLI,BR,BNE,BE,BLT,BGE,BWL,RET,JMP,PUSH,POP,RETI,ENAI,DISI,STF,LDF},            % if you want to add more keywords to the set
  keywordstyle=\color{blue},       % keyword style
  language={samurai},                % the language of the code
	tabsize=4
}

\newcommand*{\FrameXStart}{0.0}% in cm
\newcommand*{\FrameXEnd}{2.0}% in cm
\newcommand*{\FrameYClip}{0.0}% shift in cm of the frame, both from top and bottom
\newcommand*{\FrameTextShift}{-1.5}% shift in cm from the top
\newcommand*{\PageNumberLocation}{4.5}% in cm, from bottom

\newcommand*{\FrameTitle}{SAMURAI: Programmers Guide}%{\ChapterName: \SectionName}%
%
%
%% Style for frame
\tikzset{Frame Style/.style={fill=gray, draw=none, fill opacity=0.5, shade, top color=gray, bottom color=black}}
\tikzset{Title Style/.style={scale=2.75, text=white, rotate=-90, anchor=west}}
\tikzset{Title Top Shift/.style={xshift=\Midpoint cm, yshift=\FrameTextShift cm}}
\tikzset{Title Bottom Shift/.style={xshift=-\Midpoint cm, yshift=\FrameTextShift cm}}

%% Style for page numbers in frame
\tikzset{Page Number Style/.style={scale=2.0, text=white, draw=none}}
\tikzset{Page Number Odd Shift/.style={ xshift= \Midpoint cm, yshift=\PageNumberLocation cm}}
\tikzset{Page Number Even Shift/.style={xshift=-\Midpoint cm, yshift=\PageNumberLocation cm}}
%
\pgfmathsetmacro{\Midpoint}{0.5*(\FrameXStart+\FrameXEnd)}
\newcommand{\MyGraphicLogo}{% For imported graphic logo
\begin{tikzpicture}[remember picture,overlay]
\checkoddpage
\ifoddpage
    \draw [Frame Style] 
        ($(current page.north east)+(-\FrameXStart cm,-\FrameYClip cm)$) rectangle
        ($(current page.south east)+(-\FrameXEnd   cm, \FrameYClip cm)$);
    \node [Title Bottom Shift, Title Style] at (current page.north east) {\FrameTitle};
    \node [Page Number Even Shift, Page Number Style] at (current page.south east) {\thepage};
\else 
    \draw [Frame Style] 
            ($(current page.north west)+(\FrameXStart cm,-\FrameYClip cm)$) rectangle
            ($(current page.south west)+(\FrameXEnd   cm, \FrameYClip cm)$);
    \node [Title Top Shift, Title Style] at (current page.north west) {\FrameTitle};
    \node [Page Number Odd Shift, Page Number Style] at (current page.south west) {\thepage};
\fi 
\end{tikzpicture}
}
%
\SetBgContents{\MyGraphicLogo}% Select frame to be drawn

\SetBgPosition{current page.north west}% Select location
\SetBgOpacity{1.0}% Select opacity
\SetBgAngle{0.0}% Select roation of logo
\SetBgScale{1.0}% Select scale factor of logo
%\pagenumbering{gobble}
\pagestyle{empty}
