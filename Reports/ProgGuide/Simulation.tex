%  Simulation.tex
%  Document created by seblovett on hind.ecs.soton.ac.uk
%  Date created: Thu 27 Mar 2014 10:13:53 GMT
%  <+Last Edited: Sun 27 Apr 2014 16:22:02 BST by seblovett on seblovett-Ubuntu +>

\section{Simulation}

\subsection{Running the simulations}
%How to run for each of the behavioural, extracted and mixed

Before the simulator is invoked, the assembler should be run. 
This is discussed in section~\ref{sect:runningassembler}. 
It can be done from within the programs directory (\texttt{~/design/fcde/verilog/programs}) by running, for example, \texttt{assemble.py multiply}

The script ``simulate'' is an executable shell script. 
It is run from the terminal in the directory \texttt{~/design/fcde/verilog}. 
This supports running simulations of a full verilog model, cross simulation and a fully extracted simulation. 
Usage is as follows:\\
\begin{center}
\texttt{./simulate type program [definitions]}
\end{center}

The `type' can be one of the following: \textit{behavioural, mixed, extracted}.
`Program' is a relative path to the assembled hex file, usually located in the programs folder. 
Extra definitions can also be included to set the switch value or serial data input.

The serial data file used is located in the programs directory. 
This is a hex file with white space separated values of the form `` time data''. 
The data is then sent at the time to the processor by the serial module. 
An example serial data hex file is shown in listing~\ref{lst:serialdata}.

\lstinputlisting[label=lst:serialdata,caption={Example serial data file}]{../../Design/Implementation/programs/serial_data.hex}

Below is a complete list of commands to run all programs on all versions of the processor.
`Number' is a user defined decimal value to set the switches.
\begin{itemize}
\item \texttt{./assembler/assemble.py programs/multiply.asm \\ ./simulate behavioural programs/multiply.hex +define+switch\_value=number}
\item \texttt{./assembler/assemble.py programs/multiply.asm \\ ./simulate mixed programs/multiply.hex +define+switch\_value=number}
\item \texttt{./assembler/assemble.py programs/multiply.asm \\ ./simulate extracted programs/multiply.hex +define+switch\_value=number}\\

\item \texttt{./assembler/assemble.py programs/random.asm \\ ./simulate behavioural programs/random.hex +define+switch\_value=number}
\item \texttt{./assembler/assemble.py programs/random.asm \\ ./simulate mixed programs/random.hex +define+switch\_value=number}
\item \texttt{./assembler/assemble.py programs/random.asm \\ ./simulate extracted programs/random.hex +define+switch\_value=number}\\

\item \texttt{./assembler/assemble.py programs/factorial.asm \\ ./simulate behavioural programs/factorial.hex +define+switch\_value=number}
\item \texttt{./assembler/assemble.py programs/factorial.asm \\ ./simulate mixed programs/factorial.hex +define+switch\_value=number}
\item \texttt{./assembler/assemble.py programs/factorial.asm \\ ./simulate extracted programs/factorial.hex +define+switch\_value=number}\\

\item \texttt{./assembler/assemble.py programs/interrupt.asm \\ ./simulate behavioural programs/interrupt.hex programs/serial\_data.hex}
\item \texttt{./assembler/assemble.py programs/interrupt.asm \\ ./simulate mixed programs/interrupt.hex programs/serial\_data.hex}
\item \texttt{./assembler/assemble.py programs/interrupt.asm \\ ./simulate extracted programs/interrupt.hex programs/serial\_data.hex}
\end{itemize}

A scan path simulation can also be run.
This is done by running \texttt{ncverilog -sv +gui +ncaccess+r stimulus.sv  opcodes.svh cpu.sv} for a GUI or \texttt{ncverilog -sv stimulus.sv  opcodes.svh cpu.sv -exit} for a command line simulation.
This test pulses a signal on the SDI line, and verifies a pulse is seen on the output. 
The clock cycles, and therefore the number of registers, are counted and reported upon success of the simulation.

The number of clock cycles for each program to fully run is shown in table~\ref{tab:runtimes}. 
Factorial run time is given for an input of 8 and is the worst case. 
Interrupt is dependant on the serial data input and the time is given for the serial data file mentioned above.

\begin{table}
\centering
\caption{Clock cycles required for each program to run}
\label{tab:runtimes}
\begin{tabular}{|c|c|}
Program & Clock Cycles \\ \hline
Multiply	& 900	\\
Factorial	& 6000	\\
Random		& 	\\
Interrupt	& 30000	\\ \hline
\end{tabular}
\end{table}

A dissembler is also implemented in System Verilog to aid debugging.
It is an ASCII formatted array implemented at the top level of the simulation. 
It is capable of reading the instruction register with in the design, and reconstructing the assembly language of the instruction and is supported in behavioural, mixed and extracted simulations.
It will show the opcode, register addresses and immediate values.
It is automatically included by the TCL script.
The TCL script also opens a waveform window and adds important signals.
