%  RegDescription.tex
%  Document created by seblovett on hind.ecs.soton.ac.uk
%  Date created: Thu 27 Mar 2014 10:12:06 GMT
%  <+Last Edited: Mon 28 Apr 2014 13:11:45 BST by hl13g10 on octopus +>

\section{Register Description}

The processor consists of 16bit program counter and 8 general purpose registers of 16 bit each in the register file, 16 bit instruction register along with the Link register and AluOut. 

\paragraph{Register File}
The 8x16 register file has two read ports and one write ports. 
The read port takes 3-bit address inputs, RS1 and RS2, each specifying one of the 8 GPRs as source operands. 
They read the 16 bit register values onto read data outputs RD1 and RD2. 
The write port takes a 3 bit address input Rw, a 16 bit input, WD. 
The output ports of the register file serve as inputs A and B of the ALU if selected by the two multiplexers. 
By convention, register R7 is used to store the stack pointer.

\paragraph{Link Register} 
The Link Register is used to store the return address of the caller function. 
However, it is not a part of the General Purpose register file. 
For example, in this processor the RET instruction performs unconditional branch to the address which is stored in the link register (LR).

\paragraph{Program Counter} 
The Program Counter (PC) is a 16bit register wherein the output of the PC points to the current instruction and the input of the PC shows the address of the next instruction. 
The branch instructions make use of the PC. 

\paragraph{Instruction Register} 
The instruction register of 16 bit is also a storage element that has a single read port. 
It takes a 16-bit instruction address input and reads the 16-bit data or instruction from that address onto the read data output (Rd). 

\paragraph{AluOut} 
It is a 16 bit register, AluOut stores the output of the operations performed in the ALU.

