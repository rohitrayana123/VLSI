%  RegDescription.tex
%  Document created by seblovett on hind.ecs.soton.ac.uk
%  Date created: Thu 27 Mar 2014 10:12:06 GMT
%  <+Last Edited: Mon 28 Apr 2014 21:24:47 BST by seblovett on seblovett-Ubuntu +>

\subsection{Register Description}

The \samurai{} processor has twelve registers in total, all are 16 bits wide.
These is a program counter, instruction register, link register, ALU output register and 8 general purpose registers.
Each register is described below, along with any conventions.  

\paragraph{General Purpose Registers}
The register block consists of eight General Purpose Registers (GPRs). 
There is no dummy register. 
By convention, Register 7 is used as the stack pointer. 
It is used by stack and interrupt instructions such as push, pop, store and load flags and return from interrupt.  


\paragraph{Link Register} 
The Link Register is used to store the return address of the caller function. 
However, it is not a part of the General Purpose register file. 
The link register is used by the branch with link and return from subroutine instructions. 
In these, the program counter is stored to or set by the link register. 
The link register can also be pushed or popped to/from the stack.
%For example, in this processor the RET instruction performs unconditional branch to the address which is stored in the link register (LR).

\paragraph{Program Counter} 
The program counter is used to access the current instruction. 
It can be set by the result of an ALU operation, the link register, a value on the stack or a predefined constant used for interrupts.
Branch instructions are the main modifier of the program counter.
By default, all instructions increment the Program Counter by one to progress the operation of the program. 
This is not an addressable register and it's functionality is utilised by the control unit only. 

\paragraph{Instruction Register} 
The instruction register contains the currently executed instruction. 
This can only be set from the main memory by use of the Program Counter as the address to main memory.
It is not addressable and it's function is utilised by the control unit.

\paragraph{AluOut} 
The AluOut register is used to hold a value on the output of the Alu. 
It is used by memory access instructions and is not addressable. 
