\definecolor{mygreen}{rgb}{0,0.6,0}
\lstset{ %
  backgroundcolor=\color{white},   % choose the background color; you must add \usepackage{color} or \usepackage{xcolor}
  basicstyle=\footnotesize,        % the size of the fonts that are used for the code
  breakatwhitespace=false,         % sets if automatic breaks should only happen at whitespace
  breaklines=true,                 % sets automatic line breaking
  captionpos=b,                    % sets the caption-position to bottom
  commentstyle=\color{mygreen},    % comment style
  deletekeywords={...},            % if you want to delete keywords from the given language
  escapeinside={\%*}{*)},          % if you want to add LaTeX within your code
  extendedchars=true,              % lets you use non-ASCII characters; for 8-bits encodings only, does not work with UTF-8
  frame=single,                    % adds a frame around the code
  keepspaces=true,                 % keeps spaces in text, useful for keeping indentation of code (possibly needs columns=flexible)
  keywordstyle=\color{black},       % keyword style
  language=C,                 % the language of the code
  morekeywords={*,...},            % if you want to add more keywords to the set
  numbers=left,                    % where to put the line-numbers; possible values are (none, left, right)
  numbersep=5pt,                   % how far the line-numbers are from the code
  numberstyle=\tiny\color{black}, % the style that is used for the line-numbers
  rulecolor=\color{black},         % if not set, the frame-color may be changed on line-breaks within not-black text (e.g. comments (green here))
  showspaces=false,                % show spaces everywhere adding particular underscores; it overrides 'showstringspaces'
  showstringspaces=false,          % underline spaces within strings only
  showtabs=false,                  % show tabs within strings adding particular underscores
  stepnumber=1,                    % the step between two line-numbers. If it's 1, each line will be numbered
  stringstyle=\color{black},     % string literal style
  tabsize=4,                       % sets default tabsize to 2 spaces
  title=\lstname                   % show the filename of files included with \lstinputlisting; also try caption instead of title
}


\section{Programs}
Every example program in this section uses R7 as a stack pointer which is initialised to the by the program to 0x07D0 using the LUI and LLI instructions.
It is possible a stack is not required in which case no initialisation is needed and R7 can be used as a general purpose register. 

\subsection{Multiply}
The code for the multiply program is held in appendix~\ref{sec:multiply} listing~\ref{lst:multiply.asm}.
Capable of performing multiplication of two unsigned eight bit numbers to produce a single sixteen bit output.
The eight bit numbers are read from the sixteen input switches and then split in to lower and upper bytes which are then multiplied.
The resulting sixteen bit word is placed upon the LEDs before reaching a terminating loop.
A shift and add algorithm is implemented and a trade of between code size and execution time is made by loop unrolling the sixteen stages.

The subroutine is described as C in listing~\ref{lst:shiftAndAdd.c}. 
Immediately the twooperands are compared against 0xFF00. 
This checks the input parameters are only 8 bits long otherwise 0 is returned as 16 bits is the maximum length output.
In every loop the state of the lowest bit in the multiplier byte controls the accumulator. 
The multiplier is shifted right and the quotient is shifted left but no shifts occur in the last stage.
Equation~\ref{eqn:multi} formally describes the result of algorithm.

\lstinputlisting[label=lst:shiftAndAdd.c,caption=shiftAndAdd.c]{pseudocode/shiftAndAdd.c}

\begin{equation}
   A = M \times Q = \sum_{i=0}^{7} 2^i M_i Q\:\:where\:\:M_i \in \{0,1\}
   \label{eqn:multi}
\end{equation}

\subsection{Factorial}
The code for the factorial program is held in appendix~\ref{sec:factorial} listing~\ref{lst:factorial.asm}.

\subsection{Random}
The code for the random program is held in appendix~\ref{sec:random} listing~\ref{lst:random.asm}.

\subsection{Interrupt}
The code for the interrupt program is held in appendix~\ref{sec:interrupt} listing~\ref{lst:interrupt.asm}.

