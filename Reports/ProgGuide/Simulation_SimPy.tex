%  Simulation.tex
%  Document created by seblovett on hind.ecs.soton.ac.uk
%  Date created: Thu 27 Mar 2014 10:13:53 GMT
%  <+Last Edited: Fri 02 May 2014 10:16:37 BST by seblovett on seblovett-Ubuntu +>

\section{Simulation}

\subsection{Running the simulations}

%How to run for each of the behavioural, extracted and mixed

%What it does:

%Invokes assembler (if app) and simulation from anywhere. 
A python script, sim.py, was written to automatically invoke the assembler and simulator. 
The passed program is only assembled if the file exists with an extension of \texttt{.asm}. 
This allows for raw hex to be passed to the simulator where necessary. 
If a \texttt{.hex} file is passed, and a \texttt{.asm} file exists of the same name, the assembler will be invoked.
The sim.py script is configured to be run from within the assembler directory.
%The sim.py script is designed to be put on the user's path, allowing for the invocation of the assembler and simulator from anywhere. 

The usage for the script is:
\begin{center}
\texttt{sim.py [-t type] [-m module.sv / -p program.asm ] [ -s switchvalue ] [ -gdS ] [+define+extra\_definitions]}
\end{center}

%Supports all simualtions
All simulation types are supported. 
As well as full system simulations, the sim.py script also allows for other testbenches to be run.
All stimulus files are maintained in a directory and the testbenches can be run on verilog or magic modules. 
Where a Magic design is to be simulated, the script automatically extracts the netlist. 
This is done to prevent the Magic design and netlist being inconsistent. 
%Allows test benches to be run on both verilog and Magic modules. 

%Usage
The sim.py script provides a help prompt when run with \texttt{-h} or \texttt{--help} arguments. 
The help prompt is also displayed when incorrect arguments are supplied. 
The full help prompt is show in listing~\ref{lst:simpyhelp}. 

By default, the graphical user interface is not invoked. 
This can be done with the \texttt{-g} or \texttt{--gui} tags. 
A debug option, \texttt{-d}, exists when the user wants to get the majority of the simulation command, but modify it slightly. 

The program and module options should never be defined at the same time. 
One of them, however, should be. 
The program option is assembled, if necessary, and defined in the simulation command. 
The module option checks for the testbench file (identified by \textit{module}\_stim.sv) within the verification folder. 
The testbench is then used as the top level module. 

The type of simulation can be any of the folders in the verilog directory, for example \textit{behavioural}, \textit{mixed} or \textit{extracted}. 
A special type, \textit{magic} can be used.
When this is done so, the magic folder, \texttt{~/design/fcde/magic/design}, is checked for the module given. 
Type \textit{magic} and a program is equivalent to an extracted type simulation and is treated as such. 
The type is also given as a definition to the simulator, allowing reuse of test benches. 

The value of the switches can be easily defined by using the \texttt{-s} tag. 
The value given after this option is then passed to the simulator as a definition. 
If other definitions are required (for example, the serial data file), they can be defined, in full, in the trailing arguments. 
All trailing arguments are appended to the simulation command, allowing for the user to customise the invocation beyond the scope of the script. 

A scan path simulation can also be run.
This is done by running \texttt{./sim.py -S} and allows the same use described above for invoking the GUI.
If the \texttt{-S} option is defined, any program or module also given is ignored.
The scan path test pulses a signal on the SDI line, and verifies a pulse is seen on the output. 
The clock cycles, and therefore the number of registers, are counted and reported upon success of the simulation.

\lstinputlisting[label=lst:simpyhelp,caption={Help prompt for the sim.py script.}]{simpy.txt}

Finally, a \texttt{-H} or \texttt{--home} tag exists to override the default expected location. 
The script expects to be in the assembler directory within the verilog folder. 
If an absolute path is passed, the script will use this as the base directory with the \textit{behavioural}, \textit{mixed} and \textit{extracted} folders in.
A relative path can be passed. 
This should be the path from the folder the script is run from to the verilog directory. 

\subsection{Serial Data}

The serial data file used is located in the programs directory. 
This is a hex file with white space separated values of the form \textit{``time data''}.
Both values are given as hexadecimal numbers. 
The data is then sent at the time to the processor by the serial module. 
An example serial data hex file is shown in listing~\ref{lst:serialdata}.

\lstinputlisting[label=lst:serialdata,caption={Example serial data file}]{../../Design/Implementation/programs/serial_data.hex}

\subsection{Run Time}
The number of clock cycles for each program to fully run is shown in table~\ref{tab:runtimes}. 
Factorial run time is given for an input of 8 and is the worst case. 
Random is the time taken to compute a new value of the pseudo-random sequence. 
Interrupt is dependant on the serial data input and the time is given for the serial data file mentioned above.

\begin{table}
\centering
\caption{Clock cycles required for each program to run}
\label{tab:runtimes}
\begin{tabular}{|c|c|} \hline
Program & Clock Cycles \\ \hline
Multiply	& 1,100	\\
Factorial	& 3,300	\\
Random		& 2,500	\\
Interrupt	& 30,000	\\ \hline
\end{tabular}
\end{table}

\subsection{Simulation}

A dissembler is also implemented in System Verilog to aid debugging.
It is an ASCII formatted array implemented at the top level of the simulation. 
It is capable of reading the instruction register with in the design, and reconstructing the assembly language of the instruction and is supported in behavioural, mixed and extracted simulations.
It will show the opcode, register addresses and immediate values.
It is automatically included by the TCL script.
The TCL script also opens a waveform window and adds important signals.
