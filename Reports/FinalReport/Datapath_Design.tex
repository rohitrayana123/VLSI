%  Datapath_Design.tex
%  Document created by seblovett on seblovett-Ubuntu
%  Date created: Thu 17 Apr 2014 15:02:16 BST
%  <+Last Edited: Mon 05 May 2014 13:59:36 BST by seblovett on seblovett-Ubuntu +>


\section{Datapath}


Design of whole module, including circuit diagram
The datapath consists of arrays of submodules. 
The interrupt constant is also included in the datapath design.
The main submodule is the datapath bit slice. 

The program counter, register and ALU slices are made grouped together into one bitslice.
This module also implements some multiplexors for the Write Data selection and the ALU operand selection.
Other signals are extended to the edges or routed to the relevant locations. 

A ``seventeenth'' slice is also made. 
This includes the two decoders for the registers and the ALU.
The register selection multiplexors are implemented in this module as well.
An ALU override was also implemented in this slice.
The series of multiplexors are added to give the controller the ability to conduct an increment / decrement operation. 
This override was added after the initial design to be able to increment the stack pointer irrespectively of the current operation.
This was needed to implement the interrupt support to be able to save the Program Counter to the stack.

The final module needed is a right end module.
This has the right cell buffer, along with the ALU output register and tristate buffer.
It also contains an extra tristate buffer to connect the status register (from the controller) to the system bus to allow the storage of the flags.
There are two versions of this module: one to output an input signal on the system bus, and one to drive the remaining 12 bits of the system bus with a 0. 
This module also contains the memory enable tristate buffer.

\todo[inline]{Circuit diagrams for datapath slice and slice17?}

The full datapath is made up of the following modules:
\begin{enumerate}
\item Datapath Slice
\item Instruction Register modules
\item Left cell buffer
\item LUI / LLI cells
\item Right end module
\item Slice seventeen decoder
\end{enumerate}

%Design of block,
The block is constructed as seen in Figure~\ref{fig:datapath:block}
The tie high and tie low cell are used to set the constant address for the interrupt service routine.
The instruction register output is routed to both the bottom of the module to be connected to the controller and to the top of the datapath for the decoders and register selects.
The data in and out connections are routed too the right hand side of the datapath module to make an easy connection to the pad ring. 
The majority of the control signals are available on the bottom of the datapath, although some, such as the ALU override and register select signals, are only available at the top. 
This was done so that the controller would sit beneath the datapath and require little routing of the control signals.

\begin{figure}
\includegraphics[width=\textwidth]{../../eagle/Datapath/datapath.png}
\caption{Module layout to make the full datapath.}
\label{fig:datapath:block}
\end{figure}

