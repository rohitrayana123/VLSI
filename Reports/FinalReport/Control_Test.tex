%  Control_Test.tex
%  <+Last Edited: Mon 12 May 2014 14:34:59 BST by seblovett on seblovett-Ubuntu +>

\section{Controller}

State machine activity is recreated in the testbench which is verified using observable control unit outputs.
This assumes a known state on reset and state transitions are done by manipulating the inputs as described in section~\ref{sec:controller}.  
The test then applies to both the behavioural model and the netlist extracted from synthesis.
An umbrella test is used to verify no output of the controller is ever unknown during any state.
A series of test vectors are applied to the module, an example is described in Listing~\ref{lst:ControlVector.v}.

\lstinputlisting[style=sverilog,label=lst:ControlVector.v,caption=Control unit test vector.]{Code/ControlVector.v}


Activity in the control unit is dependent on the upper byte of data in the instruction register which is broken down into the $5$ bit Opcode and the $3$ bit branch condition code.  
The instruction register is populated in the fetch stage, which is always the same for control signals, so a function \texttt{DoFetch()} is used to update the instruction register with a different value for testing.
Inside the function the substate transitions are made ending with \textbf{cycle0} on exit.
Memory access violations are tested using the assertions applied in the \textbf{demux\_bus.sv} file along with the entire system as a larger framework is required.

After each fetch the variable length execute stage is tested.
The output is verified using the function \texttt{SyncOutput()}. 
This probes the datapath as such to perform the required operations.
State transitions occur outside the function.

As the interrupt functionality is a lower priority no explicit testing of entry was performed at this level.
This is proves difficult because of the nature of real-time systems.
Operations used to support the interrupts can be tested using the existing method. 
