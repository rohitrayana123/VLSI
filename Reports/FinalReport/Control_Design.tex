\section{Controller}

The controller is a Mealy machine used to sequence operations performed on the datapath.
Figure~\ref{fig:ControlBlock} contains all inputs, outputs and internal registers.
There are $7$ labelled inputs including two $4$ bit buses and one $8$ bit bus.
There are $26$ labelled outputs including five $4$ bit buses and one $3$ bit bus.
Typedefs, defined in a global file, are used for some single bit outputs and all bus outputs to keep decoding consistent in control and the datapath. 


\begin{figure}[ht]
   \centering
    \includegraphics[width = 0.8\textwidth]{ControlPinout.pdf}
		\caption{Inputs, outputs and internal registers of the control block.}% \todo[inline]{Maybe change to IEEE symbols if we have time, AJR: we still have the eagle d-types but I think it would look a bit messy} }
   \label{fig:ControlBlock}
\end{figure}

Three main states exist to service the fetch, execute and interrupt stages.
Five sub states are used within the main states to further coordinate operation.
The ASM chart in figure~\ref{fig:MainStateASM} describes state changes using the sub state, opcode and interrupt request signal.   
When \textbf{nReset} is asserted the machine will return to the first cycle of the fetch state.


\begin{figure}[ht]
   \centering
    \includegraphics[width = 0.8\textwidth]{MainStateASM.pdf}
		\caption{ASM chart of controller main states.}% \todo[inline]{Maybe change to IEEE symbols if we have time, AJR: we still have the eagle d-types but I think it would look a bit messy} }
   \label{fig:MainStateASM}
\end{figure}








\subsection{Fetch}





\subsection{Execute}




\subsection{Interrupt}


Implementation of interrupts (flags, enable\dots)


\subsection{Synthesis}

The controller was synthesised using Cadence Encounter. 
This provided a gate level netlist which was then used for place and route.
Both Magic and L-Edit place and route tools were used. 

The Magic place and route had a considerably larger routing channel than L-Edit.
The cells were all placed in one long row. 
This was suboptimal as it resulted in a lot of dead space within the floorplan.

The L-Edit place and route provided far better results. 
The datapath and controller were approximately the same width. 
The I/Os were able to be positioned around the module to reduce the overall wiring needed in the full floor plan. 
Most of the datapath control signals were routed to the top of the module to reduce the wiring channel needed between them.
All the memory access signals were routed to the left of the controller to make easier connections to the pad ring. 
The controller was also connected to the low nibble of the input data. 
This was connected at the bottom of the controller to reduce wiring needed. 
Finally, decoder signals and flag inputs were routed to the right of the control. 


