%  Conclusion.tex
%  Document created by seblovett on seblovett-Ubuntu
%  Date created: Thu 17 Apr 2014 15:25:42 BST
%  <+Last Edited: Sun 11 May 2014 10:55:58 BST by seblovett on seblovett-Ubuntu +>


\chapter{Conclusion}
%\incomplete{Conclusion}
\review{Conclusion}
\todo[inline, color=green]{MW:Ive expanded alot on what was here for more overview status of system and some evaluative conclusions}

The processor designed is a 16 bit RISC architecture which has a fully completed layout and is design rule checked. 
With added novel functionality including; multi-bit shifting, stack-specific operation \texttt{PUSH} and \texttt{POP}, varied length immediate value support and 8 GPRs with dedicated link register and program counter. 
Each module has been individually tested and verified.
However, there is a discrepancy between the behavioural model and the extracted layout.
In the individual modular test sequences, this discrepancy was not noticed. 
It is not known if the bug exists in the layout or the behavioural model. 

Appendix~\ref{ch:pm} discusses the workings of the group and the project management.
This is rounded off with individual reflections from group members.
Appendix~\ref{ch:dol} shows the overall Division of Labour within the group for the entire project.

Given the size of the group, the level of functionality gained and the progress made is higher than the original expectations of the group. 
However this has lead to a less optimized design due to the limited manpower available. 
As with interrupt support being an additional aspect not forming a part of the initial design. 
It also became apparent from the ``Balloon Debate'' that our design focussed more advanced functionality than operating speed and physical size which should have had greater consideration. 

Overall, the project has been successful in producing a functional general purpose microprocessor.  
Whilst supporting material such as a symbolic assembler has been produced to a good standard to aid any programmer with using this system. 