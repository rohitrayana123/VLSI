%  CPU_Test.tex
%  <+Last Edited: Sun 11 May 2014 10:42:42 BST by seblovett on seblovett-Ubuntu +>

\section{CPU}


A full system white noise test was devised. 
This consisted of filling the program memory with random data. 
Two simulations are then run: one with the behavioural model and one on the extracted netlist. 
If the two models are identical, then the contents of the memory and the registers should match at the end of the simulations.

A few issues were encountered in the implementation of the test.
Firstly, memory operations caused issues. 
As the memory map contains undefined areas, load operations could possibly load invalid memory into the design.
Due to the Von-Neuman architecture, this invalid memory could be stored back to the program memory and cause the processor to crash. 
Invalid memory cannot be verified, resulting in the test failing. 
For this reason, all memory operations were removed from the test data.
Similarly, jumps were also removed to prevent the program jumping to an invalid memory location. 

Interrupt instructions were also removed due to the memory returning issue with RETI, STF, LDF, and the lack of verification for the enable and disable interrupt instructions. 
Finally, due to the unpredictable nature for some flags after logic operations, any arithmetic instructions with flag dependencies (ADCI, ADCIB etc.) are also removed. 
The removed operations are replaced with ADDIB R0 127 to save repeatedly checking the data. 

%Verilog was implemented to stop the simulation when the program counter reach $0x7FF$. 
%The contents of the 8 general purpose registers were then saved to a file. 
The code implemented in SystemVerilog to save the registers is shown in Listing~\ref{lst:wn:verilog}.
The Program Counter is probed depending on which simulation is being run, and checked on the rising edge of each clock pulse. %On each clock, it is checked.
If the Program Counter is greater than or equal to $0x07FF$ (the end of valid memory), the simulation is terminated.
The values of the registers are saved to a file. 
%\todo[inline]{Include verilog listing?}

\noindent A python script was written to do the following:
\begin{enumerate}
\item Generate 2048 words of random data
\item Remove any invalid instructions (as described above)
\item Write operations to a hex file
\item Run Behavioural simulation
\item Run Extracted simulation
\item Compare the saved files
\end{enumerate}

The comparison checks the final register values and indicates a pass or fail. 
A pass is issued if all registers match. 
The test on the processor showed that there is an inconsistency between the two processors. 
The output of the test is shown in Listing~\ref{lst:wn:output}.
However, due to the use of random data, this is not a repeatable test and it is difficult to debug the issues. 

\todo[inline,color=red]{BUG: When I include the definitions.tex and change the style of the verilog listing, some contention happens with the xcolor package and breaks things.}
\inote{maybe move this listing to appendix - MW:Agreed, also may look better with the black border used in programmers guide}
\lstinputlisting[style=sverilog,firstline=184,label=lst:wn:verilog,caption=SystemVerilog Whitenoise Test]{../../Design/Implementation/verilog/behavioural/monitor.sv}

\todo[inline, color=green]{MW:Center this listing}
\begin{lstlisting}[label=lst:wn:output,caption={Output of the white noise test}]
Reg	B	E	P/F
---------------------------
0	007f	007f	P
1	5504	0073	F
2	faff	faff	P
3	581a	581a	P
4	aafc	aafc	P
5	04d4	04d4	P
6	7cff	0001	F
7	faf4	faf4	P
White Noise Test Failed.
\end{lstlisting}

%\todo[inline]{include a shot of the test results}
