%  PC_Design.tex
%  Document created by seblovett on seblovett-Ubuntu
%  Date created: Thu 17 Apr 2014 15:01:32 BST
%  <+Last Edited: Sun 11 May 2014 15:17:40 BST by seblovett on seblovett-Ubuntu +>

\section{Program Counter and Link Register}


%Design of whole module, including circuit diagram
The program counter and link register were implemented in one module.
The module has two load enable registers. 
The data source logic and an adder for incrementing the program counter is also implemented in this module.

The program counter sources are:
\begin{enumerate}
\item Incremented program counter - for normal sequential operation
\item Link Register - for returning from a subroutine call
\item ALU output - for a jump to a register contents
\item System bus - for returning the program counter from the stack after an interrupt service routine.
\item ISR Location - a constant defined for the location of the interrupt service routine in memory.
\end{enumerate}

The link register has two inputs:
\begin{enumerate}
\item Incremented program counter - for storing the return address when executing a function call
\item System Bus - for retrieving a return address from the stack 
\end{enumerate}

%Use of hierarchy / blocks - i.e. bit sliced, decoder
This module is made up of sixteen identical bitslices. 
All control signals are connected to the controller and so no decoder is needed in this module.
The circuit diagram is shown in Figure~\ref{fig:pc:circuit}.
%Design of slice,
The slice is implemented to match the circuit diagram. 
%Multiplexors are used instead of tristate buffers as there is little wiring involved. 
Multiplexors are used instead of tristate buffers since there are few data sources used which are not significantly distributed around the datapath to require a bus. 
It also removes the need for a decoder.
%\todo[inline, color=green]{MW:Some of this is surely implied, could you not say more? @seblovett; HSL: It's a very simple module, not much to say here really. MW:Guess its hard to expand}

\begin{figure}
\includegraphics[width=\textwidth]{../../eagle/PcBlock/PcBlock_slice.png}
\caption{Circuit Diagram for Program Counter and Link Register block slice.}
\label{fig:pc:circuit}
\end{figure}

%Design of block,
The module is made up of an array of slices. 
The control signals are propagated throughout the module vertically. 
The system bus connection has also been added and the carry out and the second input to the half adder must be connected between the slices.
